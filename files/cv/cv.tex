\documentclass{article}
\usepackage{hyperref}

%% \usepackage{html,htmllist}

%
% dvips -o cv.ps curric.dvi
% gv cv.ps

\newcommand{\REF}[2]{\begin{flushleft} {\bf #1} \\
                                                                           #2
                                         \end{flushleft}
                    }


\newcommand{\job}[4]{\begin{flushleft}
                                        \begin{tabular}{ll}
                                         #1-- & #3 \\
                     #2 & #4 \\
                    \end{tabular}
                                        \end{flushleft}
                    }
\newcommand{\award}[4]{\job{#1}{#2}{#3}{#4}}

\begin{document}


{
\begin{flushleft}
{\Huge{Curriculum Vitae}} \\
\hspace{.1in}
\hrule
\end{flushleft}
}

%% \begin{htmlonly}
%% \htmladdnormallink{Download Postscript File.}
%%                                {cv.ps}
%% \end{htmlonly}


%% \section*{Personal Data}

\begin{description}
\item [Name:] Leonardo de Moura
\item [Address:] 18530 NE 53rd CT, Redmond, WA 98052, USA
\item [Phone:] (425) 647-1990
\item [Email:] leodemoura0@gmail.com
\item [Homepage:] \url{http://leodemoura.github.io}
\item [Github:] \url{https://github.com/leodemoura}
\item [LinkedIn:] \url{https://www.linkedin.com/in/leonardo-de-moura-26a27b5/}
\item [Nationality:] American and Brazilian
\end{description}

\section*{Experience and Job History}

\job{04/2023}{}{Senior Principal Applied Scientist at Amazon Web Services, Redmond, WA, USA}{}

\job{01/2013}{03/2023}{Senior Principal Researcher at Microsoft, Redmond, WA, USA}{}

\job{01/2010}{01/2013}{Principal Researcher at Microsoft, Redmond, WA, USA}{}

\job{08/2006}{01/2010}{Senior Researcher at Microsoft, Redmond, WA, USA}{}

\job{02/2001}{08/2006}{Computer Scientist at SRI International, Menlo Park, CA, USA}{}

\job{04/2000}{12/2000}{Computer Engineer at Advus, S\~{a}o Paulo, Brazil}{}

\job{08/1994}{04/2000}{Research Assistant at the Software Engineering Laboratory}{of PUC-Rio}

\job{06/1998}{12/1998}{Visiting Researcher at Semantic Designs, Austin, TX, USA}{}

\section*{Projects}

I am the main architect and leader of the following projects.

\begin{itemize}
\item {\em The Lean Proof Assistant and Programming Language}, \url{http://leanprover.github.io}.
Lean is implemented in Lean itself and is fully extensible: users can modify and extend the parser, elaborator,
tactics, decision procedures, pretty printer, and code generator.
Lean is also an efficient functional programming language based on a novel programming paradigm called functional but in-place.
The project has been featured in many popular science magazines, including
{\bf Nature}~\footnote{\url{https://www.nature.com/articles/d41586-023-00487-2}}~\footnote{\url{https://www.nature.com/articles/d41586-021-01627-2}},
  {\bf Wired}~\footnote{\url{https://www.wired.com/story/the-effort-to-build-the-mathematical-library-of-the-future/}},
  {\bf BigThink}~\footnote{\url{https://bigthink.com/the-future/artificial-intelligence-replace-mathematicians/}},
  and {\bf New York Times}~\footnote{\url{https://www.nytimes.com/2023/07/02/science/ai-mathematics-machine-learning.html}}.
  Many additional links to articles and interviews are available at at \url{https://leanprover.github.io/links/}.

\item {\em Z3} is an efficient satisfiability modulo theories (SMT) solver, \url{http://github.com/z3prover/z3}.
  Z3 is used in automated testing, software and hardware verification, optimization, and constraint solving applications.
  The project has received numerous awards including the {\em Programming Languages Software Award} from ACM SIGPLAN.
  Z3 is widely used in industry (e.g., Microsoft, AWS, Apple, Meta, Google, to cite a few).

\item {\em SAL} stands for Symbolic Analysis Laboratory (\url{https://sal.csl.sri.com/}). It is a framework for combining different tools for
 model checking, theorem proving, and abstraction toward the calculation of properties (symbolic analysis) of transition systems. A
 key part of the SAL framework is an intermediate language for describing transition systems, and model checking tools.

\end{itemize}

\section*{Research Interests}

Theorem Proving, Model Checkers, Software/Hardware Verification, Decision Procedures, Programming Languages, Static Analysis.

\section*{Awards}

\award{2021}{ }{Distinguished paper award at PLDI}{“Perceus: Garbage Free Reference Counting with Reuse”.}
\award{2021}{ }{CAV Award}{for pioneering contributions to the foundations of the theory and practice\\ & of satisfiability modulo theories (SMT).}
\award{2019}{ }{Herbrand Award}{for numerous and important contributions to SMT solving, including\\ & its theory, implementation, and application to a wide range of \\ & academic and industrial needs.}
\award{2018}{ }{ETAPS 2018 Test of Time Award}{for the paper {\em Z3: An Efficient SMT Solver}}
\award{2017}{ }{Skolem Award}
      {for the paper ``Efficient E-Matching for SMT Solvers''. \\
       & The Skolem award is given to the papers that have passed the test of time \\
       & by being a most influential in the field of automated deduction.}
\award{2015}{ }{Programming Languages Software Award for Z3 from ACM SIGPLAN.}{}
\award{2014}{ }{TACAS Conference Award. }  { Most influential tool paper in the first 20 years of TACAS.}
\award{2010}{ }{Haifa Verification Conference Award. } { The HVC award is given to the most influential work in the last five \\
                   & years in the scope of software and hardware verification and testing. }
\award{2007}{ }{Microsoft Gold Star (for the Z3 theorem prover)}{Microsoft}
\award{2005}{ }{SRI Focus Award (Outstanding Employee)}{SRI International}
\award{2000}{ }{Second Prize in the ACM'2000 Student Research Contest}{Association of Computing Machinery (ACM)}

\section*{Education}
\begin{tabular}{ll}
04/2000 & Ph.D. in Computer Science, \\
  & Thesis topic: ``Automating the Generation of Program Analysis \\
  & and Verification Tools'' \\
  & Pontifical Catholic University of Rio de Janeiro (PUC-Rio), Brazil \\
03/1996 & M.Sc. in Computer Science, \\
        & Thesis topic: ``Visual Development Environments'' \\
        & Pontifical Catholic University of Rio de Janeiro (PUC-Rio), Brazil \\
01/1994 & Computer Engineer, \\
        & Pontifical Catholic University of Rio de Janeiro (PUC-Rio), Brazil \\
\end{tabular}

\section*{Publications}

Google Scholar: \url{https://scholar.google.com/citations?user=CwazDKgAAAAJ&hl=en}.

\begin{enumerate}

\item S. Ullrich and L. de Moura,
      {\em ‘do’ unchained: Embracing local imperativity in a purely functional language},
      Proc. ACM Program. Lang., ICFP, 2022.

\item L. de Moura and S. Ullrich,
      {\em The Lean 4 theorem prover and programming language},
      28th International Conference on Automated Deduction, 2021.

\item A. Reinking, N. Xie, L. de Moura, and D. Leijen,
      {\em Perceus: Garbage free reference counting with reuse},
      In Proceedings of the 42nd ACM SIGPLAN International Conference on Programming Language Design and Implementation, PLDI, 2021. ({\em distinguished paper award}).

\item S. Ullrich and L. de Moura,
      {\em Beyond notations: Hygienic macro expansion for theorem proving languages},
      10th International Joint Conference in Automated Reasoning (IJCAR), 2020.

\item S. Ullrich and L. de Moura,
      {\em Counting immutable beans: Reference counting optimized for purely functional programming},
      31st Symposium on Implementation and Application of Functional Languages (IFP), 2019.

\item D. Selsam, M. Lamm, B. Bünz, P. Liang, L. de Moura and D. Dill,
      {\em Learning a SAT Solver from Single-Bit Supervision},
      International Conference on Learning Representations (ICLR), 2019.

\item G. Ebner, S. Ullrich, J. Roesch, J. Avigad and L. de Moura,
      {\em A Metaprogramming Framework for Formal Verification},
      Proc. ACM Program. Lang., ICFP, August 2017.

\item D. Selsam and L. de Moura,
      {\em Congruence Closure in Intensional Type Theory},
8th International Joint Conference in Automated Reasoning (IJCAR), 2016.

\item R. Lewis and L. de Moura,
      {\em Automation and Computation in the Lean Theorem Prover},
      International Conference on Artificial Intelligence and Theorem Proving (AITP), 2016

\item J. Avigad, L. de Moura and S. Kong.
      {\em Theorem Proving in Lean}, 2015.

\item L. de Moura, S. Kong, J. Avigad, F. van Doorn and J. von Raumer,
      {\em  The Lean Theorem Prover},
      25th International Conference on Automated Deduction, 2015.

\item C. Barrett, L. de Moura and P. Fontaine,
  {\em Proofs in Satisfiability Modulo Theories},
  Mathematical Logic and Foundations. College Publications, London, UK, 2015.

\item A. Reynolds, C. Tinelli and L. de Moura,
  {\em Finding Conflicting Instances of Quantified Formulas in SMT},
  14th International Conference on Formal Methods in Computer-Aided Design, 2014.

\item D. Jovanovi\'{c}, C. Barrett, and L. de Moura,
  {\em The design and implementation of the model constructing satisfiability calculus},
13th International Conference on Formal Methods in Computer-Aided Design, 2013.

\item L. de Moura and G. O. Passmore,
  {\em Computation in real closed infinitesimal and transcendental extensions of the rationals},
24th International Conference on Automated Deduction, 2013.

\item L. de Moura and G. O. Passmore,
{\em The Strategy Challenge in SMT Solving},
Automated Reasoning and Mathematics: Essays in Memory of William W. McCune,
LNAI 7788, 2013.

\item L. de Moura, D. Jovanovi\'{c},
      {\em A Model-Constructing Satisfiability Calculus},
14th International Conference on Verification, Model Checking, and Abstract Interpretation (VMCAI) 2013.

\item D. Jovanovi\'{c} and L. de Moura,
      {\em Cutting to the chase solving linear integer arithmetic},
Journal of Automated Reasoning, 2013 ({\em submitted}).

\item G. Passmore, L. C. Paulson, L. de Moura,
{\em Real algebraic strategies for MetiTarski proofs},
11th International Conference, AISC 2012, 19th Symposium, Calculemus 2012.

\item D. Jovanovi\'{c}, L. de Moura,
      {\em Solving nonlinear arithmetic},
6th International Joint Conference in Automated Reasoning (IJCAR) 2012.

\item D. Jovanovi\'{c}, L. de Moura,
      {\em Solving nonlinear arithmetic},
Technical Report MSR-TR-2012-20, Microsoft Research, 2012.

\item C. Barrett, M. Deters, L. de Moura, A. Oliveras, and A. Stump,
      {\em 6 Years of SMT-COMP},
Journal of Automated Reasoning, 2012.

\item N. Bjorner, and L. de Moura,
{\em Tractability and Modern Satisfiability Modulo Theories Solvers},
Handbook of Tractability, Cambridge University Press, 2012.

\item K. Hoder, N. Bjorner, and L. de Moura,
      {\em muZ - an efficient engine for fixed points with constraints},
Computer Aided Verification (CAV) 2011.

\item L. de Moura and N. Bjorner,
      {\em Satisfiability modulo theories: introduction and applications},
Communications of the ACM, (CACM) 2011.

\item D. Jovanovi\'{c} and L. de Moura,
      {\em Cutting to the chase solving linear integer arithmetic},
23rd International Conference on Automated Deduction (CADE), 2011.

\item M. P. Bonacina, C. Lynch, and L. de Moura,
      {\em On deciding satisfiability by theorem proving with speculative inferences},
Journal of Automated Reasoning, 2011.

\item M. Veanes, N. Bjorner and L. de Moura,
{\em Symbolic Automata Constraint Solving},
International Conference on Logic programming and automated reasoning (LPAR), 2010.

\item C. Wintersteiger, Y. Hamadi and L. de Moura,
{\em Efficiently Solving Quantified Bit-Vector Formula},
International Conference on Formal Methods in Computer-Aided Design (FMCAD), 2010.

\item L. de Moura and N. Bjorner,
{\em Bugs, Moles and Skeletons: Symbolic Reasoning for Software Development},
International Joint Conference on Automated Reasoning (IJCAR), 2010.

\item N. Bjorner and L. de Moura,
{\em TAPAS Theory Combinations and Practical Applications},
invited paper at FORMATS 2009.

\item L. de Moura and N. Bjorner,
{\em Generalized and Efficient Array Decision Procedures},
International Conference on Formal Methods in Computer-Aided Design (FMCAD), 2009.

\item L. de Moura and N. Bjorner,
{\em Satisfiability Modulo Theories: An Appetizer},
invited paper to SBMF 2009.

\item G. O. Passmore and L. de Moura,
{\em Superfluous S-polynomials in Strategy-Independent Grobner Bases},
11th International Symposium on Symbolic and Numeric Algorithms for Scientific Computing (SYNASC), 2009.

\item L. de Moura and G. O. Passmore,
{\em On Locally Minimal Nullstellensatz Proofs},
International Workshop on Satisfiability Modulo Theories (SMT), 2009.

\item G. O. Passmore and L. de Moura,
{\em Universality of Polynomial Positivity and a Variant of Hilbert's 17th Problem},
ADDCT'09.

\item L. de Moura and N. Bjorner,
{\em $Z3^{10}$: Applications, Enablers, Challenges and Directions},
invited paper to CFV 2009.

\item M. P. Bonacina, C. Lynch and L. de Moura,
{\em On deciding satisfiability by DPLL(Gamma+T) and unsound theorem proving},
22nd International Conference on Automated Deduction (CADE-22), 2009.

\item Y. Ge and L. de Moura,
{\em Complete instantiation for quantified SMT formulas},
International Conference on Computer Aided Verification (CAV 2009).

\item C. Wintersteiger, Y. Hamadi and L. de Moura,
{\em A Concurrent Portfolio Approach to SMT Solving},
International Conference on Computer Aided Verification (CAV 2009).

\item R. Piskac, L. de Moura and N. Bjorner,
{\em Deciding Effectively Propositional Logic with Equality}
Technical Report: MSR-TR-2008-181.

\item N. Bjorner, B. Dutertre and L. de Moura
{\em Accelerating Lemma Learning using Joins - DPPL(Join)},
International Conference on Logic programming and automated reasoning (LPAR), 2008.

\item L. de Moura and  N. Bjorner,
{\em Proofs and Refutations, and Z3},
IWIL 2008.

\item N. Bjorner, L. de Moura and N. Tillmann,
{\em Satisfiability Modulo Bit-precise Theories for Program Exploration},
Invited workshop paper, CFV 2008.

\item L. de Moura and N. Bjorner,
{\em Deciding Effectively Propositional Logic using DPLL and substitution sets},
International Joint Conference on Automated Reasoning (IJCAR), 2008.

\item L. de Moura, N. Bjorner,
{\em Engineering DPLL(T) + Saturation},
International Joint Conference on Automated Reasoning (IJCAR), Sydney, Australia, 2008.

\item L. de Moura and N. Bjorner,
{\em Z3: An Efficient SMT Solver},
International Conference on Tools and Algorithms for the Construction and Analysis of Systems (TACAS), 2008.

\item L. de Moura and N. Bjorner, {\em Relevancy Propagation}, MSR Technical Note, 2007.

\item L. de Moura and N. Bjorner, {\em Efficient E-matching for SMT solvers}, International Conference on Automated Deduction (CADE), 2007.

\item L. de Moura and N. Bjorner, {\em Model-based Theory Combination}, Workshop on Satisfiability Modulo Theories (SMT), 2007.

\item C. Barrett, L. de Moura and A. Stump,
{\em Design and Results of the Second Satisfiability Modulo Theories Competition (SMT-COMP 2006)}, Journal of Formal Methods in System Design, 2007.

\item L. de Moura, B. Dutertre and N. Shankar,
{\em A Tutorial on Satisfiability Modulo Theories}, Conference on Computer Aided Verification (CAV), 2007.

\item B. Dutertre and L. de Moura,
{\em A Fast Linear-Arithmetic Solver for DPLL(T)}
18th International Conference on Computer Aided Verification (CAV'06).

\item C. Barrett, L. de Moura and A. Stump,
{\em Design and Results of the 1st Satisfiability Modulo Theories Competition (SMT-COMP 2005)}
Journal of Automated Reasoning (JAR), 2006.

\item C. Barrett, L. de Moura and A. Stump,
{\em SMT-COMP: Satisfiability Modulo Theories Competition}
17th International Conference on Computer Aided Verification (CAV'05).

\item G. Hamon, L. de Moura and J. Rushby,
{\em Generating Efficient Test Sets with a Model Checker},
The Second IEEE International Conference on Software Engineering and Formal Methods (SEFM'04).

\item L. de Moura, H. Rue\ss\ and N. Shankar,
{\em Justifying Equality},
Second Workshop on Pragmatics of Decision Procedures in Automated Reasoning (PDPAR'04).

\item L. de Moura, S. Owre, H. Rue\ss, J. Rushby, N. Shankar, M. Sorea and A. Tiwari,
{\em SAL 2},
16th International Conference on Computer Aided Verification (CAV'04).

\item L. de Moura and H. Rue\ss,
{\em An Experimental Evaluation of Ground Decision Procedures},
16th International Conference on Computer Aided Verification (CAV'04).

\item L. de Moura, H. Rue\ss, N. Shankar and J. Rushby,
{\em The ICS decision procedures for embedded deduction},
Second International Joint Conference on Automated Reasoning (IJCAR'04).

\item H. Rue\ss\ and L. de Moura,
{\em From Simulation to Verification (and Back)}
Proceedings of the 2003 Winter Simulation Conference.

\item L. de Moura, H. Rue\ss, J. Rushby and N. Shankar,
{\em Embedded Deduction with ICS},
Presented at the third High Confidence Software and Systems Conference, 2003.

\item L. de Moura, H. Rue\ss\ and M. Sorea,
{\em Bounded Model Checking and Induction: From Refutation to Verification},
15th International Conference on Computer Aided Verification (CAV'03).

\item L. de Moura, H. Rue\ss\ and M. Sorea,
{\em Lazy Theorem Proving for Bounded Model Checking over Infinite Domains},
International Conference on Automated Deduction (CADE'02).

\item L. de Moura, C.J. P. de Lucena and E.H. Haeusler,
{\em Analysis of Parallel Programs},
Eletronic Notes in Theoretical Computer Science, 2002.

\item L. de Moura and H. Rue\ss,
{\em Lemmas on Demand for Satisfiability Solvers},
Fifth  International Symposium on the Theory and Applications of Satisfiability Testing (SAT), 2002.

\item L. de Moura,
{\em Semantic-Directed Generation of Program Analysis and Verification Tools},
Second Prize in the ACM'2000 Student Research Contest, Austin, Texas, 2000.

\item L. de Moura, C. J. P. de Lucena and E. H. Hausler,
{\em Analysis of Parallel Programs},
Brazilian Symposium of Programming Languages (SBLP), 2000.

\item L. de Moura, C. J. P. de Lucena and E. H. Hausler,
{\em A Modular Implementation of Action Notation},
International Workshop on Action Semantics and Related Frameworks, 2000.

\item M. F. Fontoura, C. Braga, L. de Moura and C. J. P. de Lucena,
{\em Using Domain Specific Languages to Instantiate Object-Oriented Frameworks},
IEE Proceedings - Software, 147(4), 2000.

\item M. F. Fontoura, L. de Moura, S. Crespo and C. J. P. de Lucena,
{\em ALADIN: An Architecture for Learningware Application Design and Instantiation},
World Wide Web WWW Baltzer Science, Bussum, Holand, 2000.

\item I. D. Baxter, A. Yahin, S. Nedunuri, and L. de Moura,
{\em Lowering Maintenance Costs by Code Clone Removal},
12th International Software Quality Week, 1999.

\item L. de Moura, C. J. P. de Lucena and A. von Staa,
{\em The Spider Environment},
Software Practice \& Experience, 29(2), 99-124, 1999.

\item I. D. Baxter, A. Yahin, L. de Moura, M. Sant'Anna and L. Bier,
{\em Clone Detection Using Abstract Syntax Trees},
Proc. of the International Conference on Software Maintenance'98, 1998, IEEE Press.

\item L. de Moura and C. J. P. de Lucena,
{\em An Introduction To The Spider Visual Programming Environment},
Brazilian Symposium of Software Engineering (SBES), 1997.

\item H. Fuks and L. de Moura,
{\em Supporting Team Collaboration},
SIGOIS Bulletin 16, New York, pp.64-68, 1995.

\item H. Fuks and L. de Moura,
{\em A Document Based Approach for Cooperation},
Journal of the Brazilian Computer Society, V1, N1, pp 36-45, July 1994.

\item L. de Moura and R. R. dos Santos,
{\em Critical Exponents for Site-Bond Correlated Percolation},
Phys. Review B 45, 1023, 1992.
\end{enumerate}

\section*{Patents}
\begin{itemize}
\item L. de Moura and N. Bjorner, \emph{Matching based pattern inference for SMT solvers}, US Patent 9,489,221.
\item L. de Moura and N. Bjorner, \emph{Relevancy propagation for efficient theory combination}, US Patent 8,140,459.
\item L. de Moura and N. Bjorner, \emph{E-matching for SMT solvers}, US Patent 8,103,674.
\item L. de Moura and N. Bjorner, \emph{Model-based theory combination}, US Patent 7,925,476.
\item J. Rushby, L. de Moura, G. Hamon, \emph{Formal methods for test case generation}, US Patent 7,865,339.
\item L. de Moura and H. Rue\ss, \emph{Method for combining decision procedures with satisfiability solvers}, US Patent 7,653,520.
\end{itemize}

\section*{Professional Activities}
\begin{itemize}
\item Chair of the 13th International Conference on Interactive Theorem Proving (ITP), 2022.
\item Member of the Program Committee of the 23rd International Conference on Verification, Model Checking, and Abstract Interpretation (VMCAI), 2021.
\item Member of the Program Committee of the International Conference on Formal Methods in Computer-Aided Design (FMCAD), 2021.
\item Member of the Program Committee of the 12th International Conference on Interactive Theorem Proving (ITP), 2021.
\item Member of the Program Committee of the 10th International Joint Conference on Automated Reasoning (IJCAR), 2020.
\item Member of the Program Committee of the 23rd Brazilian Symposium on Formal Methods (SBMF), 2020.
\item Member of the Program Committee of the 28th International Conference on Automated Reasoning with Analytic Tableaux and Related Methods (Tableaux), 2019.
\item Member of the Program Committee of the International Conference on Tools and Algorithms for the Construction and Analysis of Systems (TACAS), 2019.
\item Member of the Program Committee of the 10th International Conference on Interactive Theorem Proving (ITP), 2019.
\item Member of the Program Committee of the 25th International Conference on Types for Proofs and Programs (TYPES), 2018.
\item Member of the Program Committee of the 9th International Joint Conference on Automated Reasoning (IJCAR), 2018.
\item Member of the Program Committee of the 9th International Conference on Interactive Theorem Proving (ITP), 2018.
\item Chair of 26th International Conference on Automated Deduction, 2017.
\item Member of the Program Committee of the 24th International Conference on Types for Proofs and Programs (TYPES), 2017.
\item Member of the Program Committee of the 26th International Conference on Automated Reasoning with Analytic Tableaux and Related Methods, 2017.
\item Chair of the Calculemus track at Conference on Intelligent Mathematics, 2016.
\item Member of the Program Committee of the International Conference on Formal Methods in Computer-Aided Design (FMCAD), 2016.
\item Member of the PhD Committee for Soonho Kong, Carnegie Mellon University, 2015.
\item Member of the Program Committee of the NASA Formal Methods Symposium (NFM), 2015.
\item Member of the Program Committee of the International Conference on Satisfiability (SAT), 2015.
\item Member of the Masters thesis committee for Robert Lewis, Carnegie Mellon University, 2014.
\item Member of the Program Committee of the International Conference on Automated Deduction (CADE, 2014).
\item Member of the Program Committee of the International Conference on Automated Deduction (CADE, 2013).
\item Member of the Program Committee of the International Conference on Tools and Algorithms for the Construction and Analysis of Systems (TACAS), 2013.
\item Member of the Program Committee of the International Conference on Satisfiability (SAT), 2013.
\item Member of the PhD committee for Chantal Keller, \'{E}cole Polytechnique, 2013.
\item Member of the Program Committee of the 5th NASA Formal Methods Symposium (NFM), 2013.
\item Chair of the 16th Brazilian Symposium on Formal Methods (SBMF), 2013.
\item Member of the PhD committee for Dejan Jovanovi\'{c}, New York University, 2012.
\item Member of the Program Committee of the International Conference on Satisfiability (SAT), 2012.
\item Member of the Program Committee of the International Symposium on Formal Methods (FM), 2012.
\item Member of the Program Committee of the International Conference on Verified Software: Theories, Tools, and Experiments (VSTTE), 2012
\item Member of the Program Committee of the International Conference on Automated Deduction (CADE, 2011).
\item Member of the Program Committee of the Workshop on Satisfiability Modulo Theories (SMT), 2011.
\item Member of the PhD committee for Christoph Wintersteiger, ETH Zurich, Switzerland, 2011.
\item Member of the Program Committee of the Symposium on Logic in Computer Science (LICS), 2011.
\item Member of the Program Committee of the International Conference on Satisfiability (SAT), 2011.
\item Member of the Steering Committee of the Workshop on Satisfiability Modulo Theories (SMT), 2009-2011.
\item Member of the PhD committee for Alberto Griggio, University of Trento, Italy, 2010.
\item Member of the Program Committee of the International Conference on Tools and Algorithms for the Construction and Analysis of Systems (TACAS), 2010.
\item Member of the PhD committee for Yeting Ge, New York University, 2009.
\item Member of the Program Committee of the International Conference on Formal Methods in Computer-Aided Design (FMCAD), 2009.
\item Member of the Program Committee of the Workshop on Satisfiability Modulo Theories (SMT), 2009.
\item Member of the Program Committee of the Workshop on Automated Formal Methods (AFM), 2009.
\item Member of the Program Committee of the Workshop on Automated Formal Methods (AFM), 2008.
\item Member of the Program Committee of the International Conference on Frontiers of Combining Systems (FroCoS), 2008.
\item Chair of the Workshop on Satisfiability Modulo Theories (SMT), 2008.
\item Member of the Program Committee of the International Conference on Satisfiability (SAT), 2008.
\item Member of the Program Committee of the Workshop on Bit-Precise Reasoning (BPR), 2008.
\item Member of the Program Committee of the Workshop on Automated Formal Methods (AFM), 2007.
\item Member of the Program Committee of the Workshop on Satisfiability Modulo Theories (SMT), 2007.
\item Member of the Program Committee of the International Conference on Satisfiability (SAT), 2007.
\item Organizer of the 2nd Satisfiability Modulo Theories Competition (SMT-COMP), 2006.
\item Member of the Program Committee of the International Conference on Formal Methods in Computer-Aided Design (FMCAD), 2006
\item Member of the Program Committee of Pragmatics of Decision Procedures in Automated Reasoning (PDPAR), 2006.
\item Tutorial Chair of the International Conference on Formal Methods in Computer-Aided Design (FMCAD), 2006.
\item Organizer of the 1st Satisfiability Modulo Theories Competition (SMT-COMP), 2005.
\end{itemize}

\section*{References}

\REF{Dr. Thomas Ball}
    {Microsoft Research, WA, USA \\
     email: tball@microsoft.com \\
     homepage:   \url{https://www.microsoft.com/en-us/research/people/tball/}
}

\REF{Prof. Jeremy Avigad}
    {Carnegie Mellon University, PA, USA \\
     email: avigad@cmu.edu \\
     homepage: \url{http://www.andrew.cmu.edu/user/avigad/}
    }

\REF{Dr. Natarajan Shankar}
    {SRI International, CA, USA \\
    email: shankar@csl.sri.com \\
    homepage: \url{http://www.csl.sri.com/\~shankar} \\
    }

\end{document}
